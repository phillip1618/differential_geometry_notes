\documentclass[12pt]{article}

\usepackage{amsmath}

\usepackage{graphicx}

\usepackage{hyperref}

\usepackage[utf8]{inputenc}

\title{Introduction to Differential Geometry Notes}

\author{Phillip Kim}

\date{2023-05-10}

\begin{document}

\maketitle

\section{What is a curve?}

\emph{\bfseries Def'n 1.1.1} \\
Parameterized curve in \( \mathcal{R}^n \) is map \( \gamma(t):(\alpha, \beta) \longrightarrow \mathcal{R}^n \) for some \( \alpha, \beta \) with \( -\infty \leq \alpha < \beta < \infty \)
\\
\\
{\bfseries Level curves} in \( \mathcal{R}^n \) \\
i.e. \( y^2 - x^2 = 0 \leftarrow \) parabola in \( \mathcal{R}^2 \) \\
The level curve above is at "level" 0. In general, we could have a level curve at level 'c' \( f(x, y) = c\)
\\
\\
{\bfseries smooth function} \( f: (\alpha, \beta) \longrightarrow \mathcal{R}^n \) is said to be smooth if derivative \( \frac{d^n f}{d t^n} \) exists \( \forall n \geq 1 \) and \( t \in (\alpha, \beta) \)
\\
\\
\emph{\bfseries Def'n 1.1.5} \\
If \( \gamma \) is a parameterized curve, first derivative \( \overset{.}{\gamma} \) is called the tangent vector of \( \gamma \) at point \( \gamma(t) \)

\end{document}